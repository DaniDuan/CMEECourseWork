\documentclass{article}

\usepackage[english]{babel}
\usepackage[utf8]{inputenc}
\usepackage{amsmath,amssymb}
\usepackage{parskip}
\usepackage{graphicx}

% Margins
\usepackage[top=2.5cm, left=3cm, right=3cm, bottom=4.0cm]{geometry}
% Colour table cells
\usepackage[table]{xcolor}

% Get larger line spacing in table
\newcommand{\tablespace}{\\[1.25mm]}
\newcommand\Tstrut{\rule{0pt}{2.6ex}}         % = `top' strut
\newcommand\tstrut{\rule{0pt}{2.0ex}}         % = `top' strut
\newcommand\Bstrut{\rule[-0.9ex]{0pt}{0pt}}   % = `bottom' strut

%%%%%%%%%%%%%%%%%
%     Title     %
%%%%%%%%%%%%%%%%%
\title{Practical: Autocorrelation in weather}
\author{Danica Duan \\ CID 01790819}
\date{\today}

\begin{document}
\maketitle

%%%%%%%%%%%%%%%%%
%   Problem 1   %
%%%%%%%%%%%%%%%%%
\section*{Question}
Are temperatures of one year significantly correlated with the next year in Key West, Florida?

\section*{Method}
First, Pearson correlation coefficient\cite{freedman2007statistics} is selected for testing linear relationships between variables, in this case the linear relationships between temperatures of previous and successive years. 
\begin{align}
    \label{Pearson correlation} % Equation label; can be used for referencing
    r = 
    \frac{ \sum_{i=1}^{n}(x_i-\bar{x})(y_i-\bar{y}) }{%
        \sqrt{\sum_{i=1}^{n}(x_i-\bar{x})^2}\sqrt{\sum_{i=1}^{n}(y_i-\bar{y})^2}}
\end{align}
Then, the normal distribution of correlation coefficients is plotted with temperatures in randomly permuted time series.
\\
\\Finally, since measurements of climatic variables in successive time-points in a time series are not independent, an approximate p-value is calculated instead of the standard p-value from correlation coefficient calculation, by taking the percentage of correlation coefficients in randomly permuted temperatures which are greater than the originally calculated correlation coefficient. 

\section*{Result Interpretation}
The null hypothesis can be set as: Temperatures of the successive years have no correlation with the previous years. 
\\
\\The originally calculated Pearson correlation coefficient of temperatures between years is 0.326, which indicates a moderately positive correlation(Figure~\ref{fig:temperature}).
\\
\\As shown in Figure~\ref{fig:Autocorrelation}, the correlation coefficient of temperatures between successive years appears at the far right end of the distribution. And after conducting the permutation analysis for multiple times, all p-values obtained are lower than 0.001, which indicates a significantly low likelihood of randomly obtaining such correlation (r = 0.326), therefore suggesting the correlation of temperature between years is highly significant. This result can be regarded as a strong evidence to support that temperatures for successive years are significantly correlated with the previous years in the given data set.
\\
\\The correlation can be extended into further years if there is no major additional impacts from environmental changes in the following years at Key West, Florida. 
\begin{figure}[h!]
    \centering
    \includegraphics[scale=0.65]{../results/temperature.pdf}
    \caption{Temperature correlation between successive and previous years.}
    \label{fig:temperature}
\end{figure}

\begin{figure}[h!]
    \centering
    \includegraphics[scale=0.65]{../results/Autocorrelation_plot.pdf}
    \caption{Density plot for correlation coefficients after randomly permuting the time series.}
    \label{fig:Autocorrelation}
\end{figure}


\bibliographystyle{plain}
\bibliography{Autocorrelation}

\end{document}
