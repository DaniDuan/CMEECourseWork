\documentclass{article}

\usepackage[english]{babel}
\usepackage[utf8]{inputenc}
\usepackage{amsmath,amssymb}
\usepackage{parskip}
\usepackage{graphicx}

% Margins
\usepackage[top=2.5cm, left=3cm, right=3cm, bottom=4.0cm]{geometry}
% Colour table cells
\usepackage[table]{xcolor}

% Get larger line spacing in table
\newcommand{\tablespace}{\\[1.25mm]}
\newcommand\Tstrut{\rule{0pt}{2.6ex}}         % = `top' strut
\newcommand\tstrut{\rule{0pt}{2.0ex}}         % = `top' strut
\newcommand\Bstrut{\rule[-0.9ex]{0pt}{0pt}}   % = `bottom' strut

%%%%%%%%%%%%%%%%%
%     Title     %
%%%%%%%%%%%%%%%%%
\title{Practical: Autocorrelation in weather}
\author{Danica Duan \\ CID 01790819}
\date{\today}

\begin{document}
\maketitle

%%%%%%%%%%%%%%%%%
%   Problem 1   %
%%%%%%%%%%%%%%%%%
\section*{Question}
Are temperatures of one year significantly correlated with the next year?

\section*{Method}
First, Pearson correlation coefficient\cite{freedman2007statistics} was selected for testing linear relationships between variables, in this case the linear relationships between temperatures of previous and successive years. 
\begin{align}
    \label{Pearson correlation} % Equation label; can be used for referencing
    r = 
    \frac{ \sum_{i=1}^{n}(x_i-\bar{x})(y_i-\bar{y}) }{%
        \sqrt{\sum_{i=1}^{n}(x_i-\bar{x})^2}\sqrt{\sum_{i=1}^{n}(y_i-\bar{y})^2}}
\end{align}
Then, the normal distribution of correlation coefficients was plotted with randomly permuted time series.
\\
\\At last, an approximate p-value was calculated by taking the percentage of randomly permuted coefficients which are greater than the correlation coefficient between years (calculated in the first step). 

\section*{Result Interpretation}
The null hypothesis can be set as: Temperatures of the successive years have no correlation with the previous years. 
\\
\\The Pearson correlation coefficient of temperatures calculated between years was 0.326, which possibly indicates a moderately positive correlation.
\\
\\As shown in Figure~\ref{fig:Autocorrelation}, the coefficient of temperatures between successive years appeared at the far right end of the distribution. And after permuting the time series for multiple times, all p-values obtained were lower than 0.001, which indicates a significantly low likelihood of randomly obtaining such correlation (r = 0.326), therefore suggesting the correlation of temperature between years was highly significant. This result can be regarded as a strong evidence for rejecting the null hypothesis.
\\
\\Therefore in general, this result indicates the temperatures of successive years are highly significantly correlated with previous years, and the correlation is moderately positive. 
\begin{figure}[h!]
    \centering
    \includegraphics[scale=0.85]{Autocorrelation_plot.pdf}
    \caption{Density plot for correlation coefficients after randomly permuting the time series.}
    \label{fig:Autocorrelation}
\end{figure}


\bibliographystyle{plain}
\bibliography{pearson}

\end{document}
